% \documentclass[10pt,a4paper]{moderncv}                
%\usepackage[utf8]{inputenc}
% \usepackage[scale=0.85]{geometry}
\documentclass[11pt,a4paper]{moderncv}

%\usepackage[T1]{fontenc}      % Activer toutes les toucher du clavier
%\usepackage[francais]{babel}  % Langue Français

% moderncv themes
%\moderncvtheme[orange]{casual}         
\moderncvtheme[blue]{classic}        

% character encoding
\usepackage[utf8]{inputenc}                   % replace by the encoding you are using
%
% adjust the page margins
\usepackage[top=1.5cm, bottom=1cm, left=1.5cm, right=1.5cm]{geometry}
%\usepackage{hyperref}
%\hypersetup{colorlinks=false}

%\recomputelengths                             % required when changes are made to page layout lengths

\usepackage{ifpdf}
\ifpdf
	\pdfinfo {
		/Author (Nicolas REY)
		/Title (Nicolas REY - Développeur Scala / Spark - Big Data)
		% /Subject (SUBJECT)
		% /Keywords (KEYWORDS)
%		/CreationDate (D:20110109220854)
	}
\fi

%\addtolength{\oddsidemargin}{10pt}
%\addtolength{\textwidth}{0pt}
%\addtolength{\topmargin}{-05pt}
%\addtolength{\textheight}{10pt}
%\addtolength{\marginparsep}{0pt}
%\addtolength{\marginparwidth}{0pt}

\makeatletter
\renewcommand*{\bibliographyitemlabel}{\@biblabel{\arabic{enumiv}}}
\makeatother

\nopagenumbers{}	% uncomment to suppress automatic page numbering for CVs longer than one page


\firstname{Nicolas}
\familyname{Rey}
\title{Développeur Scala / Spark - Big Data}
\address{74 rue Championnet}{75018 Paris}
\mobile{+33 6 26 09 05 07}
\email{n.rey@lateral-thoughts.com}
% \extrainfo{31 ans, Célibataire, Permis voiture}


\begin{document}
\maketitle

\vspace{1cm}
\section{Formations}
\cventry{Mars. 2018}{Formation Deep Learning}{Université de Montréal}{}{Montréal}{
  \smallskip
  Cette formation d’une semaine consistait en une grosse introduction au deep learning, les concepts
mathématiques sous jacents, ainsi que plusieurs travaux pratiques.
  \bigskip
}
\cventry{Sept. 2014}{Formation Apache Spark}{Lateral Thoughts}{}{Paris}{
  \smallskip
  \begin{itemize}
    \item Spark Core
    \item Spark SQL
    \item Spark Streaming
  \end{itemize}
  \bigskip
}
\cventry{2008 - 2012}{École Centrale de Nantes - Option Informatique}{École d'ingénieur généraliste}{}{}{
  L'École Centrale de Nantes est une école d'ingénieur généraliste.
}

\vspace{0.5cm}
\section{Langues}
\cvlanguage{Français}{Langue maternelle}{}
\cvlanguage{Anglais}{Courant (ToEIC : 885)}{Travail en Anglais, stage en Inde (4 mois), Afrique du Sud (6 séjours)}
\cvlanguage{Espagnol}{Notions}{}
\cvlanguage{Russe}{Notions}{}

\vspace{0.5cm}
\section{Informatique}
\begin{tabular}{ll}
  \vspace{0.1cm}
  \hspace{0.1cm} Langages & \hspace{0.5cm} \textbf{Scala}, Java, JavaScript, Bash (bases) \\ \vspace{0.1cm}
  \hspace{0.1cm} Ecosystème Big Data & \hspace{0.5cm} \textbf{Spark}, Kafka, HDFS, Cloudera Platform \\ \vspace{0.1cm}
  \hspace{0.1cm} Ecosystème Scala & \hspace{0.5cm} Play 2!, Akka (bases) \\ \vspace{0.1cm}
  \hspace{0.1cm} Intégration & \hspace{0.5cm} SBT, Maven, Jenkins, Git \\ \vspace{0.1cm}
  \hspace{0.1cm} Bases de Données & \hspace{0.5cm} Cassandra, MongoDB \\ \vspace{0.1cm}
  \hspace{0.1cm} Administration & \hspace{0.5cm} GNU/Linux, Ansible, Docker, Vagrant \\ \vspace{0.1cm}
	\hspace{0.1cm} Cloud Computing & \hspace{0.5cm} Microsoft Azure \\ \vspace{0.1cm}
\end{tabular}



\newpage
\section{Expériences}
% \cventry{}{}{}{}{}{}
\cventry{2018 - Auj\\2012 - 2017\\6 ans}{Ingénieur Développeur Scala/Spark - Big Data}{Lateral Thoughts}{Paris}{}{
  \small Consultant spécialisé en développement \textbf{Scala}, \textbf{Spark}, et écosystèmes Big Data
  \medskip
	\begin{itemize}
		\item Conseil des clients par rapport à leurs besoins
		\item Développement de solutions avec Scala, Spark, Kafka et autres outils de l'écosystème
	\end{itemize}
	\medskip
  Lateral Thoughts est ce que l'on appelle aujourd'hui une "entreprise libérée"
	\medskip
  \begin{itemize}
    \item Pas de chef ni de hiérarchie. Complètement auto-organisée
    \item Grande coopération, émulation et entraide entre employés
    \item Une expertise en Scala, Spark, et le développement Big Data
  \end{itemize}
  \medskip
  \textbf{Environnement :} \textit{\textbf{Scala}, \textbf{Spark}, Kafka, Cassandra, Hadoop, Scrum}
  \bigskip
}
\cventry{2018 - Auj\\8 mois}{Ingénieur Développeur Scala/Spark - Big Data}{Renault Digital}{Paris}{France}{
  \small Développement d'un outil de traçabilité des pièces et coûts de véhicules avec Scala et Spark. \\
  \medskip
  \begin{itemize}
    \item Conseil sur l'architecture et les solutions techniques à mettre en œuvre
    \item Implémentation avec Scala / Spark des cas d'usage métier (volumétrie env. 3 To)
    \item Stockage des données dans HDFS, ElasticSearch et HBase
    \item Développement des APIs pour exposer nos données avec Scala et Play2!
    \item Pratique du Test Driven Development (TDD)
    \item Rôle de devops pour outiller l'équipe et automatiser ce qui pouvait l'être
  \end{itemize}
  \medskip
  \textbf{Environnement :} \textit{\textbf{Scala}, \textbf{Spark}, HBase, ElasticSearch, HDP} \\
  \textbf{Entreprise :} \textit{Renault Digital est une filiale du groupe Renault rassemblant les projets numériques}
  \bigskip
}
\cventry{2017 - 2018\\11 mois}{Ingénieur Développeur Scala/Spark - Big Data}{agileDSS}{Montréal}{Canada}{
  \small Conseiller \textbf{Scala / Spark / Big Data}. \\
	J'ai également donné deux sessions de \textbf{formation Spark Core} chez Microsoft.
	\medbreak
	Pour les clients, mon rôle consistait en :
	\medskip
	\begin{itemize}
		\item Les aider dans la définition de leur architecture Big Data
		\item Implémentation avec Scala / Spark
		\item Pratique stricte du Test Driven Development (TDD)
		\item Formation de leurs employés sur Scala, Spark, et l'écosytème Big Data
	\end{itemize}
	\medskip
  Pour l'agence, mon rôle consistait en :
	\medskip
  \begin{itemize}
    \item Participer à l'évolution de la pratique Big Data
    \item Choix des technos Big Data de référence
    \item Faire passer la plupart des entretiens techniques de recrutement
		\item Former des employés aux technos et à la pratique Big Data
  \end{itemize}
  \medskip
  J'ai principalement travaillé avec Microsoft Azure par choix des clients.
  \medbreak
  \textbf{Environnement :} \textit{\textbf{Scala}, \textbf{Spark}, Microsoft Azure} \\
	\textbf{Entreprise :} \textit{AgileDSS est une SSII spécialisée dans la BI et le conseil autour de la donnée}
  \bigskip
}
\cventry{2016 - 2017\\7 mois}{Ingénieur Développeur Scala/Spark - Big Data}{Pages Jaunes}{Paris}{}{
  \small Refonte de la solution de traitement des données d'audience avec Scala et Spark
  \medskip
  \begin{itemize}
    \item Développement de plusieurs applications d'ingestion et de valorisation de données
		\item Unification du format en Avro et Parquet
    \item Mise en place de l'usine logicielle (Jenkins, Sonatype Nexus)
    \item DevOps : déploiement avec Ansible, pipeline de jobs avec Jenkins
    \item Formation de l'équipe à Scala, Spark et aux pratiques Agile (SCRUM)
  \end{itemize}
  \medskip
  \textbf{Environnement :} \textit{\textbf{Scala}, \textbf{Spark}, HDFS, Cloudera, Ansible, Avro, Parquet} \\
  \textbf{Gestion de projet :} \textit{Agile / SCRUM} \\
  \textbf{Entreprise :} \textit{Pages Jaunes est un moteur de recherche de professionnels. Env. 5 millions d'utilisateurs quotidiens.}
  \bigskip
}
\cventry{2015 - 2016\\10 mois}{Ingénieur Développeur Scala - Big Data}{Carrefour}{Paris}{}{
  \small Développement de la plateforme data lake temps réel de Carrefour (pour la région France)
  \medskip
  \begin{itemize}
    \item Ingestion de multiples sources de données avec Kafka
    \item Traitement de la donnée avec Spark Core et Spark Streaming
    \item Normalisation des données au format avro
    \item Persistance dans HDFS et Cassandra
    \item Exposition des données par des APIs développées avec Spray
		\item Déploiement avec Ansible
    \item Plate-forme répartie sur un cluster Mesos et un cluster Cloudera
  \end{itemize}
  \medskip
  \textbf{Environnement :} \textit{\textbf{Scala}, \textbf{Spark}, Mesos, Akka, Cassandra, Kafka, HDFS, Cloudera, Ansible, Avro} \\
  \textbf{Entreprise :} \textit{Carrefour est un des leader de la grande distribution. 5600 magasins en France.}
  \bigskip
}
\cventry{2014 - 2015\\1 an}{Ingénieur Développeur Scala/Java}{Vidal}{Paris}{}{
	\small Développement logiciel pour l'industrie du médicament
	\medskip
  \begin{itemize}
    \item Développement en Scala / Akka d'un outil de pilotage de mises à jour d'un client lourd
    \item Développement d'APIs REST HATEOAS en Java pour exposer une base de données de médicaments
    \item Développement d'un projet de tests BDD en Cucumber
		\item Pratique intensive du Test Driven Development (TDD)
		\item Pair programming
		\item Code reviews
  \end{itemize}
  \medskip
  \textbf{Environnement :} \textit{Scala, Java, Akka, Guava, Spring, Hibernate, Cucumber, Github, Docker} \\
  \textbf{Gestion de projet :} \textit{Agile / Scrum} \\
  \textbf{Entreprise :} \textit{Vidal édite une base de données de médicaments. Leader français de ce domaine.}
  \bigskip
}
\cventry{2014\\9 mois}{Ingénieur Développeur Java}{SACEM}{Paris}{}{
	\small Refonte de plusieurs portails de déclaration et de paiement
	\medskip
  \begin{itemize}
    \item Serveur développé en Java avec Sprink et beaucoup de Guava
    \item Interface Web utilisant JavaScript et ReactJS
    \item Test d'interfaces avec Fluentlenium
    \item Développement avec RabbitMQ d'un outil asynchrone pour supporter l'indisponibilité de la BDD
		\item Utilisation de Gitlab et Jenkins pour la gestion du code
  \end{itemize}
  \medskip
  \textbf{Environnement Backend :} \textit{Java, Guava, Spring, Fluentlenium} \\
  \textbf{Environnement Frontend :} \textit{JavaScript, ReactJS} \\
  \textbf{Gestion de projet :} \textit{Agile / Scrum} \\
  \textbf{Entreprise :} \textit{La SACEM est la société de collecte des droits d'auteur sur la musique en France.}
  \bigskip
}
\cventry{2013\\7 mois}{Ingénieur Développeur Scala}{FONTYOU}{Paris}{}{
	\small Développement d'une web app de co-création de polices pour une startup. \\
	Rôle très polyvalent car une équipe de seulement deux développeurs.
	\medskip
  \begin{itemize}
    \item Développement des API en Scala, Play2!, MongoDB
    \item Développement de l'interface avec JavaScript, BackboneJS, CSS3
    \item Admin Sys de l'usine logicielle et de l'environnement de prod
    \item Développement d'un POC de dessin vectoriel dans le navigateur
  \end{itemize}
  \medskip
  \textbf{Environnement Backend :} \textit{\textbf{Scala}, Play2!, MongoDB, Zabbix} \\
  \textbf{Environnement Frontend :} \textit{JavaScript, BackboneJS, RequireJS, HTML5/CSS3, LessCSS} \\
  \textbf{Gestion de projet :} \textit{Agile / Scrum} \\
  \textbf{Entreprise :} \textit{Fontyou est une startup qui travaillait à l'époque à développer une plate-forme de co-création de polices de caractères.}
  \bigskip
}
\cventry{2012 - 2013\\7 mois}{Ingénieur Développeur Scala/Java}{Vidal}{Paris}{}{
	\small Développement d'un client lourd
	\medskip
  \begin{itemize}
    \item Développement d'un système de notification installé chez les clients
    \item Gestion de l'expiration et du renouvellement de la license des clients
    \item Modélisation du cycle de vie des mises à jour avec une machine à état fini implémentée avec Akka
		\item Développement d'APIs REST HATEOAS en Java pour exposer une base de données de médicaments
  \end{itemize}
  \medskip
  \textbf{Environnement :} \textit{\textbf{Scala}, Java, Akka, SWT, JavaScript, HTML/CSS} \\
  \textbf{Gestion de projet :} \textit{Agile / Scrum} \\
  \textbf{Entreprise :} \textit{Vidal édite une base de données de médicaments. Leader français de ce domaine.}
  \bigskip
}
%\cventry{Nov. 2012\\1 mois}{Ingénieur Développeur Java}{Vidal}{Paris}{}{
%        \small Développement d'APIs REST HATEOAS
%	\medskip
%  \begin{itemize}
%    \item Développement et intégration de règles métier sur des données médicales
%    \item Exposition de ces données par des APIs REST HATEOS suivant le protocole ATOM
%  \end{itemize}
%  \medskip
%  \textbf{Environnement :} \textit{Java, Spring, Hibernate, protocole ATOM. TDD avec JUnit, Mockito.} \\
%  \textbf{Gestion de projet :} \textit{Agile / Scrum} \\
%  \textbf{Entreprise :} \textit{Vidal édite une base de données de médicaments. Leader français de ce domaine.}
%  \bigskip
%}
% \cventry{}{Références sur demande}{}{}{}{}
% \cventry{Avril - Sept. 2012\\6 mois}{Stage de fin d'études - Développement Web/Mobile}{Eklaweb}{Nantes}{}{
%   Développement d'applications mobiles HTML5 avec Backbone.js.\\
%   Serveur : Scala, Unfiltered, Play2!, MongoDB, PostgreSQL.\\
%   Article sur le blog technique d'Eklaweb : "Gestion des dates en informatique".
% }
% \cventry{Mai à août 2011\\4 mois}{Stage ingénieur en développement mobile}{Amadeus}{Bangalore}{Inde}{
%   Application iPad pour valider la maturité du développement mobile HTML5.\\
%   Utilisation importante de JavaScript (jQuery), PhoneGap, HTML5, CSS3 et SQLite.\\
%   Web scraping d'un site avec un script Python.
% }
% \cventry{Février à avril 2011}{Application Android de simulation mécanique}{Institut de recherche en Génie Mécanique}{Nantes}{}{Hautes performances nécessaires. Java, OpenGL ES, Bibliothèque de calcul matriciel (EJML).}
%\cventry{}{}{}{}{}{}


\newpage
\section{Projets personnels}
\cventry{2019}{Spark Tools}{}{}{}{
  \small Développement d'un SourceManager pour abstraire la lecture/écriture de source et la création d'une table Hive. Ajoute plusieurs SaveMode très pratiques.
  \bigskip
}
\cventry{2018}{Slack Notifier}{}{}{}{
  \small Développement d'un outil de notification Slack spécialisé pour les jobs Spark.
  \bigskip
}
\cventry{2017}{Image Docker pour TaigaIO}{}{}{}{
  \small Création d'une image Docker pour le logiciel \textit{taiga.io}. Taiga est un outil simplifié de gestion de tâches orienté méthodes Agile.
  \bigskip
}
\cventry{2017}{Image Docker pour Azkaban}{}{}{}{
  \small Création d'une image Docker pour le logiciel \textit{Azkaban}. Azkban est un gestionnaire de workflow pratique développé par LinkedIn.
  \bigskip
}
\cventry{2015}{Serveur Personnel}{}{}{}{
  \small Installation d'un serveur de développement avec Ansible \\
  \textit{Ansible, Nginx, OwnCloud, GitLab, Linux Debian}
  \bigskip
}
\cventry{2015}{Ordinateur Personnel}{}{}{}{
  \small Automatisation de l'installation et de la configuration de mes ordinateurs personnels avec Ansible
}

\section{Centres d'intérêt}
\cvline{Informatique}{Logiciels libres, Big Data.}
\cvline{Loisirs}{Salsa et autres danses, Photographie, Géopolitique, Bonnes pitances...}
\cvline{Voyages}{17 ans à Libreville (Gabon), Road trips en Europe (11 pays), 4 mois en Inde, 1 an à Montréal.}

\closesection{}                   % needed to renewcommands
\renewcommand{\listitemsymbol}{-} % change the symbol for lists

\end{document}
